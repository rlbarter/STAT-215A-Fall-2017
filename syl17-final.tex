\documentclass[11pt]{article}
\usepackage{fullpage}
\usepackage{url}
\setlength{\topmargin}{-.5in}
\setlength{\evensidemargin}{0in}
\setlength{\oddsidemargin}{0in}
\setlength{\textwidth}{6.5in}
\setlength{\textheight}{9.5in}

\begin{document}
\begin{center}
{\large \bf Statistics 215A, Fall 2017}
\end{center}
{\bf Instructor:} Professor Bin Yu\\
{\bf Lectures:} T/Th: 11:00 am -12:30 pm,  344 (?) Evans \\
{\bf Discussion:} Friday: 9-11 am,  332 Evans\\

\noindent
{\bf Text books:}
\begin{itemize}
\item Draft of book ``Data Sciene in action'' by Bin Yu and Rebecca Bater.
\item Statistical models, D. Freedman (required).
\item The Elements of statistical learning, Hastie et al (recommended).
\end{itemize}


\noindent
{\bf Bin's Office Hours in 409 Evans:}  to be announced. 
\\[1ex]
{\bf Phone:} 642-2021 (Office), 642-2781 (dept, messages),
% 549-2441 (home; emergencies) 
{\bf email:} binyu@stat.berkeley.edu \\[1ex]
{\bf Comments, Suggestions, Gripes:} in person, email, 
anonymous notes in my box or under the door.  All feedback is
welcome.\\[2ex]
{\bf GSI and office hours:}  
Rebecca Barter (rebeccabarter@berkeley.edu): Office hours to be announced.
GSI will be in charge of the discussion sessions and the labs/homework.
\\[1ex]
{\bf Phone:} 
%658-2544 (home; emergencies); 
642-2781 (Stat. Dept. main no.)
\\[1ex]
{\bf Grading:}
\begin{itemize}
\item 55\% assignments (homework and labs)
\item 5\% class/discussion and participation
\item 15\% midterm (written exam in class)
\item 25\% final project
\end{itemize}
{\bf Assignments:}
There will be 4 or 5 assignments given out on Friday in the discussion session and usually
due in two weeks (there will be an announcement if otherwise). {\bf The assignments
actually require two weeks of work to satisfactorily complete. So, it is a good idea to start very early.} The assignments contain homework problems and data analysis labs. 
For the data labs, each student will produce a 12-page (maximum) report presenting a narrative that connects the motivating questions, the analysis conducted and the conclusions drawn. The reports will be made using Knitr/Sweave and the final pdf output should not contain any code whatsoever. Each report will be hosted in a github repository containing both the code and the written report.  {\bf No late assignments} will be accepted, {\em for any reason.} \\

\noindent 
{\bf Course description:}

\begin{itemize}
\item {\bf Overview}
Information technology advances have made it possible to collect 
huge amounts of data in every walk of our life and beyond. 
These vast amounts of data have enabled scientists, social
scientists, government agencies, and companies to ask increasingly complex questions 
aimed at understanding the physical and human world,
making pubic policies, and improve productivity.
However having data alone is not enough; statistics is indispensable in the process of obtaining
meaningful answers from collected data. Not only are the common statistical models incredibly 
powerful, but statistical experimental 
design itself provides principles and methods
to collect data in order to effectively address the questions asked.

The most influential contributions can be made when domain experts (scientists, for example) and
statisticians work together to brainstorm and ask questions. These domain experts not only are key to
formalizing the ideas, but they also are integral in generating the data. Engaging with the individuals 
who collected the data in the first place allows the statistician to learn about all the 
context in which the data lives, and subsequently, to conduct an effective analysis capable of actually 
answering the question being asked.

\item {\bf Collaborative learning in context}
This course will demonstrate what is like to be an applied statistician
or data scientist in today's data-rich world. We emphasize working with people
of domain expertise in order to answer questions outside of statistics using data. 
We illustrate through lectures, class discussions, data labs, and homework assignments, the many steps involved
in the iterative process of information extraction.
This process includes data exploration, 
prediction, identification of sources of randomness in data, inference, and interpretation.
Three principles of data science, predictability, stability and computability,
will be applied in the process when appropriate.

The lectures (and labs) will be based on real-data problems, and students will learn 
useful statistical concepts and methods in the contexts of these problems. 
The goal is to illustrate how judgement and common-sense
are crucial to the process of conducting data analysis and drawing conclusions. 
While the statistical techniques will be introduced through a first-principles approach, 
students will learn to develop custom techniques in less familiar situations.

The essential elements of applied statistics are captured in Bin's piece titled ``Data Wisdom'' (\url{http://www.odbms.org/2015/04/data-wisdom-for-data-science/}) and students are asked to read the piece after the first lecture.

\item {\bf Data lab format and peer-grading}
%The class format will be a combination of lecture and discussion groups.
The data labs will be done individually, except for one group lab later in the semester. 
The goal of writing the lab report is not only to gain data analysis experience, but is also 
an exercise in communication. We ask that particular attention
is given to the writing of the report, as your peers will be reading them:
so that the students can learn from one another, the labs will be peer-reviewed. 
Each student will review 2-3 labs from their peers, and will provide feedback and a grade based on several criteria
including clarity of writing, validity of analysis and informativeness of visualizations. 
The final grade of each lab will be decided by the GSI who will use the student grades as a guide.


\item {\bf Full commitment to the class is necessary}
Please be aware that this is a heavy-load class. If you are not sure that you can commit, 
please audit the class instead, since there are many students on the waitlist. 
Further, because class discussions are an integral part of the course,
registered students are required to attend all classes unless permitted
by the instructor under justifiable circumstances. {\em After the first
3 lectures, students are expected to read sections from the draft book 
BEFORE each lecture so we could devote more class time to group
discussions.} 

\item {\bf Pre-requisites}
In this class, we require knowledge of upper division mathematical
statistics and probability courses (Stat 134 and 135) at UC Berkeley.  
In terms of computing, at a minimum you should be comfortable manipulating files in Unix and
writing your own functions, manipulating and cleaning data and
creating and customizing graphics in R. Ideally students will already have a basic fluency in the 
``tidyverse'' in R as well as confident using github. While we will be providing a short introduction to
these topics in the labs, students who are entirely unfamiliar with these tools will need to put in 
some work to ensure that they meet the standards expected of the course.\\

\end{itemize}

\newpage

{\bf Tentative list of topics:}

%{\rm In addition to the topics listed below, there will be a focus on oral and written communication skills in both the labs and class discussion.}

\begin{itemize}

\item 
\textbf{Overview of the class and logistics.}
(0.5 weeks) (Aug. 24)
%Introduction of the fruitfly project.
%Basic data collection issues and basic concepts in causal inference. 
%Statistical investigation as an iterative learning process.
%The necessity of a question or a desire to understand something before a statistical investigation.
%Data collection (observational
%vs controlled studies, basic experimental design). 
%Data quality
%If time allows, other data issues (storage, conversion, quality) 
%Aug. 28, Sept. 2).
%Reading: Freedman's first chapter
%First discussion: R. 

\item 
Starting with a high-level question, discovery-driven Exploratory Data Analysis (EDA) with a stability consideration. Numerical summaries and visual descriptions of data.
(2 weeks: Aug. 29, 31, Sept. 5, 7) 

\item Prediction and assessment. Least squares for prediction. 
Stability consideration through appropriate data perturbation (1 week) (Sept. 12 \& 14 )
%Building relationships between variables.
%The task of prediction.
%Assessing uncertainty and validation: cross-validation and bootstrap.
%one lecture:  EM algorithm (bowman tutorial
%and wikipedia and its demo) and mixture of gaussian (needs to work
%out the details)

\item Sources of randomness in data: simple and cluster samplings,
randomized experiment, natural experiment (2 weeks, Sept 19, 21, 26, 28) 

\item 
Linear regression models, their justifications and their interpretations 
(1.5 weeks: Oct. 3, 5, and Oct. 6 Lab time as lecture time)

\item Bin is on travel in the week of Oct. 9.
One lecture will be for a class discussion on a research paper using
linear regression. The other lecture will be used as Lab time.

\item LS as adjustment in Neyman-Rubin model (0.5 weeks: Oct. 17)

\item 
Classification: SVM, Logistic regression, weighted LS for logistic regression
computation, and inference in Logistic regression. (1 week: Oct. 17, 19)

\item Midterm week (Oct. 24 Review; Oct. 26 Midterm)

\item Exponential family and GLMs (1 week: Oct. 31, Nov. 2)

\item Multiple hypothesis testing and FDR (0.5 weeks: Nov. 7)

\item Regularization in Regression and GLMs I: PCA-based, Ridge, James-Stein, 
PLS (0.5 weeks: Nov. 9)

\item
Regularization in Regression and GLMs II: model selection, forward selection, L2boosting, Lasso and Sparse PCA via penalized regression. Inference related
to Lasso.  (1 week: Nov. 14, 16)

\item Final project assigned on Nov. 17 in the discussion session.

\item Dimensionality reduction via random projection (Nov. 21) 

\item Advanced topics (Nov. 28, 30)

\item 
Rebecca runs lab/discussion session on Dec. 1. No in-class final exam, but there is a final project.
\end{itemize}

{\bf Final Project Due}: Dec. 8 (Friday), 5 pm.

\end{document}
